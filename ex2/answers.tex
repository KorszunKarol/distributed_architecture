\documentclass{article}
\usepackage{listings}
\usepackage{enumitem}

\begin{document}

\section*{EXERCISE 2: QUESTIONNAIRE ABOUT THREADS AND CONCURRENCY}

\subsection*{1. Advantages and Disadvantages of Parallel vs Distributed Programming}

\textbf{Parallel Programming (Shared Memory):}
\begin{itemize}
    \item Advantages:
    \begin{itemize}
        \item Lower latency due to direct memory access
        \item Simpler programming model
        \item More efficient data sharing
        \item Lower communication overhead
    \end{itemize}
    \item Disadvantages:
    \begin{itemize}
        \item Limited scalability (bound by hardware)
        \item Higher risk of race conditions
        \item Complex debugging
        \item Hardware cost for high-performance systems
    \end{itemize}
\end{itemize}

\textbf{Distributed Programming (Shared Nothing):}
\begin{itemize}
    \item Advantages:
    \begin{itemize}
        \item Better scalability
        \item Higher fault tolerance
        \item Cost-effective scaling
        \item Geographic distribution possible
    \end{itemize}
    \item Disadvantages:
    \begin{itemize}
        \item Higher latency
        \item Network dependency
        \item Complex system coordination
        \item Data consistency challenges
    \end{itemize}
\end{itemize}

\subsection*{2. Real-World Applications}

\textbf{Parallel Programming Applications:}
\begin{enumerate}
    \item Scientific simulation software (molecular dynamics)
    \item Real-time image processing in medical equipment
    \item Financial market analysis systems
\end{enumerate}

\textbf{Distributed Programming Applications:}
\begin{enumerate}
    \item Social media platforms
    \item Global content delivery networks (CDN)
    \item Distributed gaming servers
\end{enumerate}

\subsection*{6. Difference between run() and start()}

The key difference between run() and start() methods lies in their execution context:

\begin{itemize}
    \item \textbf{start()} method:
    \begin{itemize}
        \item Creates a new thread and executes run() in that thread
        \item Allows true parallel execution
        \item Can only be called once per Thread instance
    \end{itemize}
    \item \textbf{run()} method:
    \begin{itemize}
        \item Executes in the current thread when called directly
        \item Behaves like a normal method call
        \item No new thread is created
        \item Can be called multiple times
    \end{itemize}
\end{itemize}

\end{document}
